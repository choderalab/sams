% PRL look and style (easy on the eyes)
\RequirePackage[hyphens]{url}
\documentclass[aps,pre,twocolumn,nofootinbib,superscriptaddress,linenumbers,11point]{revtex4-1}
% Two-column style (for submission/review/editing)
%\documentclass[aps,prl,preprint,nofootinbib,superscriptaddress,linenumbers]{revtex4-1}

%\usepackage{palatino}

% Change to a sans serif font.
\usepackage{sourcesanspro}
\renewcommand*\familydefault{\sfdefault} %% Only if the base font of the document is to be sans serif
\usepackage[T1]{fontenc}
%\usepackage[font=sf,justification=justified]{caption}
\usepackage[font=sf]{floatrow}

% Rework captions to use sans serif font.
\makeatletter
\renewcommand\@make@capt@title[2]{%
 \@ifx@empty\float@link{\@firstofone}{\expandafter\href\expandafter{\float@link}}%
  {\textbf{#1}}\sf\@caption@fignum@sep#2\quad
}%
\makeatother

\usepackage{listings} % For code examples
\usepackage[usenames,dvipsnames,svgnames,table]{xcolor}

\usepackage{minted}

\usepackage{amsmath}
\usepackage{amssymb}
\usepackage{graphicx}
%\usepackage[mathbf,mathcal]{euler}
%\usepackage{citesort}
\usepackage{dcolumn}
\usepackage{boxedminipage}
\usepackage{verbatim}
\usepackage[colorlinks=true,citecolor=blue,linkcolor=blue]{hyperref}

\usepackage{subfigure}  % use for side-by-side figures

% The figures are in a figures/ subdirectory.
\graphicspath{{figures/}}

% italicized boldface for math (e.g. vectors)
\newcommand{\bfv}[1]{{\mbox{\boldmath{$#1$}}}}
% non-italicized boldface for math (e.g. matrices)
\newcommand{\bfm}[1]{{\bf #1}}          

%\newcommand{\bfm}[1]{{\mbox{\boldmath{$#1$}}}}
%\newcommand{\bfm}[1]{{\bf #1}}
%\newcommand{\expect}[1]{\left \langle #1 \right \rangle}                % <.> for denoting expectations over realizations of an experiment or thermal averages

% Define some useful commands we will use repeatedly.
\newcommand{\T}{\mathrm{T}}                                % T used in matrix transpose
\newcommand{\tauarrow}{\stackrel{\tau}{\rightarrow}}       % the symbol tau over a right arrow
\newcommand{\expect}[1]{\langle #1 \rangle}                % <.> for denoting expectations over realizations of an experiment or thermal averages
\newcommand{\estimator}[1]{\hat{#1}}                       % estimator for some quantity from a finite dataset.
\newcommand{\code}[1]{{\tt #1}}

% vectors
\newcommand{\x}{\bfv{x}}
\newcommand{\y}{\bfv{y}}
\newcommand{\f}{\bfv{f}}

\newcommand{\bfc}{\bfm{c}}
\newcommand{\hatf}{\hat{f}}

%\newcommand{\bTheta}{\bfm{\Theta}}
%\newcommand{\btheta}{\bfm{\theta}}
%\newcommand{\bhatf}{\bfm{\hat{f}}}
%\newcommand{\Cov}[1] {\mathrm{cov}\left( #1 \right)}
%\newcommand{\Ept}[1] {{\mathrm E}\left[ #1 \right]}
%\newcommand{\Eptk}[2] {{\mathrm E}\left[ #2 \,|\, #1\right]}
%\newcommand{\T}{\mathrm{T}}                                % T used in matrix transpose
%\newcommand{\conc}[1] {\left[ \mathrm{#1} \right]}

\DeclareMathOperator*{\argmin}{argmin}
\DeclareMathOperator*{\argmax}{argmax}
\newcommand*{\argminl}{\argmin\limits}
\newcommand*{\argmaxl}{\argmax\limits}

%%%%%%%%%%%%%%%%%%%%%%%%%%%%%%%%%%%%%%%%%%%%%%%%%%%%%%%%%%%%%%%%%%%%%%%%%%%%%%%%
% DOCUMENT
%%%%%%%%%%%%%%%%%%%%%%%%%%%%%%%%%%%%%%%%%%%%%%%%%%%%%%%%%%%%%%%%%%%%%%%%%%%%%%%%

\begin{document}

%%%%%%%%%%%%%%%%%%%%%%%%%%%%%%%%%%%%%%%%%%%%%%%%%%%%%%%%%%%%%%%%%%%%%%%%%%%%%%%%
% TITLE
%%%%%%%%%%%%%%%%%%%%%%%%%%%%%%%%%%%%%%%%%%%%%%%%%%%%%%%%%%%%%%%%%%%%%%%%%%%%%%%%

\title{Optimal expanded ensemble simulations via self-adjusted mixture sampling}

\author{John D. Chodera}
 \thanks{Corresponding author}
 \email{random@choderalab.org}
 \affiliation{Computational Biology Program, Sloan Kettering Institute, Memorial Sloan Kettering Cancer Center, New York, NY 10065}

\date{\today}

%%%%%%%%%%%%%%%%%%%%%%%%%%%%%%%%%%%%%%%%%%%%%%%%%%%%%%%%%%%%%%%%%%%%%%%%%%%%%%%%
% ABSTRACT
%%%%%%%%%%%%%%%%%%%%%%%%%%%%%%%%%%%%%%%%%%%%%%%%%%%%%%%%%%%%%%%%%%%%%%%%%%%%%%%%

\begin{abstract}


% KEYWORDS
\emph{Keywords: biomolecular simulation; molecular dynamics; expanded ensembles; self-adjusted mixture sampling}

\end{abstract}

\maketitle

%%%%%%%%%%%%%%%%%%%%%%%%%%%%%%%%%%%%%%%%%%%%%%%%%%%%%%%%%%%%%%%%%%%%%%%%%%%%%%%%
% FIGURE: ONE-COLUMN
%%%%%%%%%%%%%%%%%%%%%%%%%%%%%%%%%%%%%%%%%%%%%%%%%%%%%%%%%%%%%%%%%%%%%%%%%%%%%%%%

%\begin{figure}[tbp]
%\resizebox{0.9\textwidth}{!}{\includegraphics{toc-graphic.pdf}}
%\caption{\label{figure:example} {\bf Example figure.} 
%This is an example figure.
%Shaded regions denote 95\% confidence intervals.
%}
%\end{figure}

%%%%%%%%%%%%%%%%%%%%%%%%%%%%%%%%%%%%%%%%%%%%%%%%%%%%%%%%%%%%%%%%%%%%%%%%%%%%%%%%
% INTRODUCTION
%%%%%%%%%%%%%%%%%%%%%%%%%%%%%%%%%%%%%%%%%%%%%%%%%%%%%%%%%%%%%%%%%%%%%%%%%%%%%%%%

\section*{Introduction}
\label{section:introduction}

Wolf chartreuse beard, paleo bushwick locavore tumblr selvage health goth narwhal post-ironic meggings cronut DIY etsy. 
Tote bag viral craft beer migas, brooklyn keffiyeh shabby chic wayfarers godard scenester affogato pabst. 
Humblebrag chartreuse schlitz, post-ironic wolf ethical narwhal salvia everyday carry gastropub venmo kale chips. You probably haven't heard of them cornhole tilde readymade mixtape irony. 
Sriracha occupy yuccie, green juice roof party fap tumblr hammock mumblecore ramps pabst. Artisan listicle truffaut kogi, shabby chic kombucha distillery etsy cronut +1 pabst mustache VHS vinyl green juice. 
Before they sold out brooklyn yuccie, gluten-free sriracha lumbersexual four loko kombucha semiotics letterpress biodiesel kale chips art party normcore slow-carb.

%%%%%%%%%%%%%%%%%%%%%%%%%%%%%%%%%%%%%%%%%%%%%%%%%%%%%%%%%%%%%%%%%%%%%%%%%%%%%%%%
% INTRODUCTION
%%%%%%%%%%%%%%%%%%%%%%%%%%%%%%%%%%%%%%%%%%%%%%%%%%%%%%%%%%%%%%%%%%%%%%%%%%%%%%%%

\section*{Methods}
\label{section:methods}

\subsection*{Self-adjusted mixture sampling}

While there are many ways to determine $\mu^*$, here, we utilize a \emph{self adjusted mixture sampling} (SAMS) simulation~\cite{sams}.
In a SAMS simulation, we allow the number $n$ of neutral salt molecules to fluctuate dynamically and the relative reduced free energy $f_n \equiv - \ln Z_n$ of each is computed through a recursion strategy in which the $f_n$ are dynamically evolved during the simulation.
{\color{red}[JDC: More details on SAMS.]}
\begin{eqnarray}
Z_n &\equiv& \int dx \, e^{-\beta [U_n(x) + p V(x)]}
\end{eqnarray}
where $U_n(x)$ denotes there are exactly $n$ salt pairs in the system.

Once the relative reduced free energies $f_n$ have have been determined, we can compute the chemical potential $\mu^*$ by recognizing that
\begin{eqnarray}
\left< n \right>_{\mu, \beta, p, N} &=& \sum_{n = 0}^\infty n \, \int dx \, e^{-\beta [ U_n(x) + p V((x) + \mu n ]} \\
&=&  \sum_{n = 0}^\infty n \, e^{\beta \mu n} \int dx \, e^{-\beta [ U_n(x) + p V((x) ]} \\
&=&  \sum_{n = 0}^\infty n \, e^{\beta \mu n} e^{-f_n} \\
&=&  \sum_{n = 0}^\infty n \, e^{\beta \mu n -f_n}
\end{eqnarray}
and write $c(\mu, \beta, p, N)$ in terms of the $\{f_n\}$ and the associated average volumes $V_n \equiv \left< V \right>_{\beta, p, N, n}$
\begin{eqnarray}
c(\mu, \beta, p, N) &=& \left< \frac{n}{N_A} \right>_{\mu, \beta, p, N} / \left< V \right>_{\mu, \beta, p, N} \\
&=& \frac{\sum_{n = 0}^\infty \frac{n}{N_A} \, e^{\beta \mu n -f_n}}{\sum_{n = 0}^\infty V_n \, e^{\beta \mu n -f_n}}
\end{eqnarray}
which allows us to solve for $\mu^*$:
\begin{eqnarray}
c_0 &=& \frac{\left< \frac{n}{N_A} \right>_{\mu, \beta, p, N}}{\left< V \right>_{\mu, \beta, p, N}} \\
&=& \frac{\sum_{n = 0}^\infty \frac{n}{N_A} \, e^{\beta \mu n -f_n}}{\sum_{n = 0}^\infty V_n \, e^{\beta \mu n -f_n}}
\end{eqnarray}

%%%%%%%%%%%%%%%%%%%%%%%%%%%%%%%%%%%%%%%%%%%%%%%%%%%%%%%%%%%%%%%%%%%%%%%%%%%%%%%%
% ACKNOWLEDGMENTS
%%%%%%%%%%%%%%%%%%%%%%%%%%%%%%%%%%%%%%%%%%%%%%%%%%%%%%%%%%%%%%%%%%%%%%%%%%%%%%%%

\section*{Acknowledgments}

JDC acknowledges a Louis V.~Gerstner Young Investigator Award, NIH core grant P30-CA008748, and the Sloan Kettering Institute for funding during the course of this work.

%%%%%%%%%%%%%%%%%%%%%%%%%%%%%%%%%%%%%%%%%%%%%%%%%%%%%%%%%%%%%%%%%%%%%%%%%%%%%%%%
% BIBLIOGRAPHY
%%%%%%%%%%%%%%%%%%%%%%%%%%%%%%%%%%%%%%%%%%%%%%%%%%%%%%%%%%%%%%%%%%%%%%%%%%%%%%%%

\bibliographystyle{prsty} 
\bibliography{manuscript}

\end{document}